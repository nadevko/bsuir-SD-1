\documentclass{bsuir}

\departmentlong{инженерной психологии и эргономики}
\workcode{9}
\worktitle{Разработка, отладка и испытание приложения параллельной обработки
данных, применение мьютексов и семафоров}
\titleleft{
    Проверил:\\
    Давыдович К.И.\\
    ~
}
\titleright{
    Выполнил:\\
    Бородин А.Н.\\
    гр. 310901
}
\titlepageyear{2024}

\newfontface\liberationrm{Liberation Serif}
\newcommand{\csharp}{C{\liberationrm\#}}

\begin{document}

\maketitle
\mainmatter
\renewcommand{\thefigure}{\arabic{figure}}
\renewcommand{\thelisting}{\arabic{listing}}

\textbf{Цель}: научиться параллельно обрабатывать данные, используя классы Task
и Parallel библиотеки TPL для \csharp.

\section*{Задание}

Используя библиотеку TPL создайте длительную по времени задачу (на основе Task)
на выбор:

\begin{itemize}
  \item поиск простых чисел (желательно взять \textquote{решето Эратосфена});
  \item перемножение матриц;
  \item умножение вектора размера 100000 на число;
  \item создание множества Мандельброта;
  \item другой алгоритм.
\end{itemize}

Выведите идентификатор текущей задачи, проверьте во время выполнения "---
завершена ли задача и выведите ее статус.

Оцените производительность выполнения используя объект Stopwatch на нескольких
прогонах. Дополнительно: Для сравнения реализуйте последовательный алгоритм.

Реализуйте второй вариант этой же задачи с токеном CancellationToken и
используйте его для отмены задачи.

Создайте три задачи с возвратом результата и используйте их для выполнения
четвертой задачи. Например, расчет по формуле.

Создайте задачу продолжения (continuation task) в двух вариантах:

\begin{itemize}
  \item C ContinueWith - планировка на основе завершения множества предшествующих
        задач;
  \item На основе объекта ожидания и методов GetAwaiter(), GetResult().
\end{itemize}

Распараллельте вычисления циклов For() и ForEach() используя класс Parallel.
Например, обработку (преобразования) последовательности, генерация нескольких
массивов по 1000000 элементов, быстрая сортировка последовательности, обработка
текстов. Оцените производительность по сравнению с обычными циклами.

Используя Parallel.Invoke() распараллельте выполнение блока операторов.

Используя Класс BlockingCollection реализуйте задачу: Есть 5 поставщиков бытовой
техники, они завозят уникальные товары на склад (каждый по одному) и 10
покупателей "--- покупают все подряд, если товара нет "--- уходят. В вашей
задаче: спрос превышает предложение. Изначально склад пустой. У каждого
поставщика своя скорость завоза товара. Каждый раз при изменении состоянии
склада выводите наименования товаров на складе.

Используя async и await организуйте асинхронное выполнение метода.

\makelisting{Program.cs}

\makelisting{1.txt}[Вывод программы (Без отмены)]

\makelisting{2.txt}[Вывод программы (С отменой)]

\textbf{Вывод}: освоены классы Task и Parallel библиотеки TPL.

\end{document}
