\documentclass{bsuir}

\departmentlong{инженерной психологии и эргономики}
\workcode{8}
\worktitle{Разработка, отладка и испытание многопоточного приложения с
синхронизированными потоками}
\titleleft{
    Проверил:\\
    Давыдович К.И.\\
    ~
}
\titleright{
    Выполнил:\\
    Бородин А.Н.\\
    гр. 310901
}
\titlepageyear{2024}

\newfontface\liberationrm{Liberation Serif}
\newcommand{\csharp}{C{\liberationrm\#}}

\begin{document}

\maketitle
\mainmatter

\phantomheading*{Задание 8}

\textbf{Цель}: научиться работать с многопоточностью на языка \csharp,
используя класс Thread.

\section*{Задание 8.2}

Создайте программу, в которой два потока одновременно увеличивают значение
одной и той же переменной. Используйте механизм синхронизации для
предотвращения гонок данных.

\makelisting{Program.cs}

\makelisting{1.txt}[Вывод программы]

\textbf{Вывод}: освоена многопоточность в \csharp, создание потоков и
обработка исключений в них.

\end{document}
