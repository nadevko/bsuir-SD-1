\documentclass{bsuir}

\departmentlong{инженерной психологии и эргономики}
\workcode{3}
\worktitle{Пространства имён. Система типов.\\Class. Struct. Enum. Record}
\titlepageyear{2024}

\newfontface\liberationrm{Liberation Serif}
\newcommand{\csharp}{C{\liberationrm\#}}

\begin{document}

    \maketitle{
        Проверил:\\
        Давыдович К.И.\\
        ~
    }{
        Выполнил:\\
        Бородин А.Н.\\
        гр. 310901
    }

    \textbf{Цель}: получить навыки создания и использования собственных
    пространств имён. Изучить основные способы организации пользовательских
    типов данных.

    \section*{Задание 3.2}

    Согласно варианта описать объект с помощью класса (Учащийся). В нем должно
    быть не менее трех полей (свойств), желательно, разных типов.

    Создать класс, который будет реализовывать:

    \begin{enumerate}
        \item метод для создания массива объектов;
        \item метод для заполнения массива объектами;
        \item метод для вывода массива объектов на экран;
        \item метод для удаления экземпляра класса из массива;
        \item метод для добавления экземпляра класса в массив.
    \end{enumerate}

    Реализовать консольное меню, для проверки функционала. В качестве типа
    одного из полей класса использовать enum (например, для цвета).

    \makelisting{1.cs}

    \makelisting{1.txt}[Вывод программы]

    \textbf{Вывод}: освоена структуризация программ, принятая в языке \csharp.

\end{document}
