\documentclass{bsuir}

\departmentlong{инженерной психологии и эргономики}
\workcode{4}
\worktitle{ООП. Наследование. Абстрактные классы и члены классов. Виртуальные
члены классов. Запечатанные классы и члены классов}
\titleleft{
    Проверил:\\
    Давыдович К.И.\\
    ~
}
\titleright{
    Выполнил:\\
    Бородин А.Н.\\
    гр. 310901
}
\titlepageyear{2024}

\newfontface\liberationrm{Liberation Serif}
\newcommand{\csharp}{C{\liberationrm\#}}

\begin{document}

\maketitle
\mainmatter
\renewcommand{\thefigure}{\arabic{figure}}
\renewcommand{\thelisting}{\arabic{listing}}

\textbf{Цель}: получить навыки создания абстрактных классов и членов класса,
создания виртуальных членов классов. Изучить наследование и основные принципы
ООП.

\section*{Задание (2)}

По полученному базовому классу (из предыдущей лабораторной работы) создать
классы наследников по двум разным ветвям наследования ($B\leftarrow
P_1\leftarrow P_{11}, B\leftarrow P_2\leftarrow P_{21}$):

\begin{itemize}
    \item во всех классах должны быть свои данные (характеристики объектов);
    \item во всех классах создать конструкторы инициализации объектов для
          всех классов (не забыть про передачу параметров в конструкции
          базовых классов);
    \item остальные методы создавать по необходимости.
\end{itemize}

Нужно создать в базовом классе виртуальные функции расчёта (например, расчёт
площади фигуры и т.п.) и вывода объекта на экран (всех его параметров).
Выполнить реализацию этих виртуальных функций в классах наследниках.

В классе контейнере создать массив, состоящий из объектов базового класса.
Заполнить массив динамически создаваемыми объектами производных классов ($P_1,
P_{11}, P_2, P_{21}$). Для каждого элемента массива проверить работу виртуальных
функций.

Отладить и выполнить полученную программу. Проверить, что будет, если
вышеописанные методы не будут виртуальными.

\makelisting{Program.cs}

\makelisting{1.txt}[Вывод программы]

\textbf{Вывод}: изучены абстрактные классы и основные принципы
объетно"=ориентированного программирования в языке \csharp.

\end{document}
