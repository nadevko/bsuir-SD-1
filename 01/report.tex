\documentclass{bsuir}

\departmentlong{инженерной психологии и эргономики}
\workcode{1}
\worktitle{Разработка, отладка и испытание программ, основанных\\на операторах
ветвления и циклических алгоритмах}
\titleleft{
    Проверил:\\
    Давыдович К.И.\\
    ~
}
\titleright{
    Выполнил:\\
    Бородин А.Н.\\
    гр. 310901
}
\titlepageyear{2024}

\newfontface\liberationrm{Liberation Serif}
\newcommand{\csharp}{C{\liberationrm\#}}

\begin{document}

\maketitle
\mainmatter
\renewcommand{\thefigure}{\arabic{figure}}
\renewcommand{\thelisting}{\arabic{listing}}

\textbf{Цель}: научиться разрабатывать программы с применением ввода"=вывода
информации на экран, условного оператора, оператора множественного выбора и
циклов на языке программирования \csharp.

\section*{Условный оператор if else (2)}

Товар стоит a руб. b коп. За него заплатили c руб. d коп. Сколько сдачи
требуется получить? Вводятся 4 числа: a, b, c и d. Необходимо вывести 2 числа в
виде: «Сдача составляет e рублей, f копеек».

\makelisting{Program1.cs}

\makelisting{1_1.txt}[Вывод программы (нулевой остаток)]

\makelisting{1_2.txt}[Вывод программы (копейки)]

\makelisting{1_2.txt}[Вывод программы (задолженность)]

\section*{Оператор выбора, оператор множественного выбора (2)}

Пользователь вводит номер дня недели. Программа должна вывести его наименование
на экран.

\makelisting{Program2.cs}

\makelisting{2_1.txt}[Вывод программы (положительное число)]

\makelisting{2_2.txt}[Вывод программы (отрицательное число)]

\makelisting{2_2.txt}[Вывод программы (неверный ввод)]

\section*{for и while (2)}

Данную задачу необходимо решить при помощи цикла for, при помощи цикла while
осуществить контроль введенных данных. Т.е. программа не заканчивает выполнение
пока не будут введены валидные данные.

Написать программу, которая возводит число в целочисленную степень. Число и
степень вводятся с клавиатуры. В степень возводить посредством цикла, Math.Pow
не использовать. Число должно быть не более 100, степень не более 10.

\makelisting{Program3.cs}

\makelisting{3_1.txt}[Вывод программы (положительное число)]

\makelisting{3_2.txt}[Вывод программы (отрицательное число)]

\makelisting{3_3.txt}[Вывод программы (неверный ввод)]

\textbf{Вывод}: освоен минимальный набор конструкции языка \csharp.

\end{document}
