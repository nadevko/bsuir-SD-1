\documentclass{bsuir}

\departmentlong{инженерной психологии и эргономики}
\workcode{2}
\worktitle{Использование алгоритмов и программ с методами}
\titlepageyear{2024}

\newfontface\liberationrm{Liberation Serif}
\newcommand{\csharp}{C{\liberationrm\#}}

\begin{document}

\maketitle{
    Проверил:\\
    Давыдович К.И.\\
    ~
}{
    Выполнил:\\
    Бородин А.Н.\\
    гр. 310901
}

\textbf{Цель}: ознакомление со структурой программ с применением
пользовательских методов, методов массивов и строк с использованием языка
программирования \csharp.

\section*{Задание 2}

Написать метод, который принимает строку любой длины и содержания. Данная строка
это пинкод. Метод возвращает true если пинкод валидный и false иначе. Пинкод
может содержать только 4 или 6 цифр.

\makelisting{1.cs}

\makelisting{1.txt}[Вывод программы (верный ввод)]

\makelisting{2.txt}[Вывод программы (неверный ввод)]

\textbf{Вывод}: освоено понятие методов и методы строк языка \csharp.

\end{document}
