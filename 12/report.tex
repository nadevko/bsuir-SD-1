\documentclass{bsuir}

\departmentlong{инженерной психологии и эргономики}
\workcode{12}
\worktitle{Разработка, отладка и испытание\\программ с использованием файлов}
\titleleft{
    Проверил:\\
    Давыдович К.И.\\
    ~
}
\titleright{
    Выполнил:\\
    Бородин А.Н.\\
    гр. 310901
}
\titlepageyear{2024}

\newfontface\liberationrm{Liberation Serif}
\newcommand{\csharp}{C{\liberationrm\#}}

\begin{document}

\maketitle
\mainmatter
\renewcommand{\thefigure}{\arabic{figure}}
\renewcommand{\thelisting}{\arabic{listing}}

\textbf{Цель}: получить навыки создания и реализации интерфейсов. Изучить
стандартные коллекции языка \csharp.

\section*{Задание 1}

Создать текстовый файл, записать в него построчно данные, которые вводит
пользователь. Окончанием ввода пусть служит пустая строка.

\makelisting{Program1.cs}

\makelisting{1.txt}[Вывод программы]

\makelisting{2.txt}[output.txt]

\makelisting{3.txt}[Вывод программы (путь не указан)]

\makelisting{4.txt}[Вывод программы неверный путь)]

\section*{Задание 2}

\makelisting{5.txt}[Вывод программы]

\makelisting{6.txt}[output.json]

\makelisting{7.txt}[Вывод программы (Путь не указан)]

\makelisting{8.txt}[Вывод программы (Неверный путь)]

Создать JSON"=файл, записать в него построчно данные, которые вводит
пользователь. Окончанием ввода пусть служит пустая строка.

\makelisting{Program2.cs}


\textbf{Вывод}: освоены интерфейсы и стандартные коллекции языка \csharp.

\end{document}
