\documentclass{bsuir}

\departmentlong{инженерной психологии и эргономики}
\workcode{10}
\worktitle{Разработка, отладка и испытание программ\\с пользовательскими диалоговыми
окнами}
\titleleft{
    Проверил:\\
    Давыдович К.И.\\
    ~
}
\titleright{
    Выполнил:\\
    Бородин А.Н.\\
    гр. 310901
}
\titlepageyear{2024}

\newfontface\liberationrm{Liberation Serif}
\newcommand{\csharp}{C{\liberationrm\#}}

\begin{document}

\maketitle
\mainmatter
\renewcommand{\thefigure}{\arabic{figure}}
\renewcommand{\thelisting}{\arabic{listing}}

\textbf{Цель}: Сформировать навыки разработки программ с использованием Windows
Forms.

\section*{Задание}

\begin{enumerate}
    \item Создайте в среде разработки MS VS проект. В качестве типа проекта
          выбрать \textquote{Windows Application} (или \textquote{Windows Forms
              Application} в зависимости от версии .NET Framework).
    \item После создания и сохранения проекта измените программу таким образом,
          чтобы координаты курсора мыши выводились в заголовке главного окна
          приложения.
    \item Добавьте текстовое поле (TextBox) в режиме разработки (для этого
          необходимо использовать панель элементов управления
          \textquote{ToolBox}). Дополните обработчик движения мыши таким
          образом, чтобы в текстовом поле отображалась сумма координат указателя
          мыши.
    \item Работающую программу необходимо представить преподавателю.
    \item После этого выполните индивидуальное задание в соответствие с
          вариантом. В каждом задании необходимо вывести значение выражения,
          предварительно введя значения переменных в соответствующие текстовые
          поля формы главного окна приложения. Результат выводится в заголовок
          окна в ответ на нажатие кнопки (кнопку также необходимо поместить на
          форму).
    \item Для выполнения индивидуального задания необходимо использовать
          математические функции, которые доступны в виде статических методов
          класса Math.
\end{enumerate}

\makeimage[Окно программы]{1.png}[width=.7\textwidth]

\makelisting{Program.cs}[Program.cs]

\makelisting{MainForm.cs}[MainForm.cs]

\makelisting{MainForm.Designer.cs}[MainForm.Designer.cs]

\makelisting{MainForm.resx}[MainForm.resx]

\textbf{Вывод}: освоена разработка программ с использованием фреймворка Windows
Forms.

\end{document}
