\documentclass{bsuir}

\departmentlong{инженерной психологии и эргономики}
\workcode{5}
\worktitle{Обобщения и шаблоны}
\titlepageyear{2024}

\newfontface\liberationrm{Liberation Serif}
\newcommand{\csharp}{C{\liberationrm\#}}

\begin{document}

    \maketitle{
        Проверил:\\
        Давыдович К.И.\\
        ~
    }{
        Выполнил:\\
        Бородин А.Н.\\
        гр. 310901
    }

    \textbf{Цель}: получить навыки создания абстрактных классов и членов класса,
    создания виртуальных членов классов. Изучить наследование и основные принципы
    ООП.

    \section*{Задание 5.2}

    Построить контейнерный шаблонный класс (очередь) операций над элементами
    данных, включающий операции:

    \begin{itemize}
        \item добавления;
        \item удаления;
        \item поиска;
        \item просмотра;
        \item сортировки элементов;
        \item перестановки элементов в обратном порядке;
        \item замены всех подобных элементов по заданному ключу;
        \item поиска максимального элемента.
    \end{itemize}

    Остальные функции добавлять по необходимости.

    \makelisting{1.cs}

    \makelisting{1.txt}[Вывод программы]

    \textbf{Вывод}: освоены шаблоны и основные принципы ООП в языке \csharp.

\end{document}
