\documentclass{bsuir}

\departmentlong{инженерной психологии и эргономики}
\workcode{6}
\worktitle{Делегаты, события и лямбды}
\titlepageyear{2024}

\newfontface\liberationrm{Liberation Serif}
\newcommand{\csharp}{C{\liberationrm\#}}

\begin{document}

    \maketitle{
        Проверил:\\
        Давыдович К.И.\\
        ~
    }{
        Выполнил:\\
        Бородин А.Н.\\
        гр. 310901
    }

    \textbf{Цель}: познакомиться с использованием делегатов в приложениях. Научится
    описывать собственные события. Познакомится с механизмом обработки событий.

    \section*{Задание 6.2}

    Для созданного в предыдущей лабораторной работе контейнерного класса
    реализовать методы, которые в качестве аргумента принимают делегат:
    
    \begin{itemize}
        \item метод сортировки;
        \item метод поиска;
        \item метод фильтрации.
    \end{itemize}

    Для каждого делегата определить один метод и одно лямбда-выражение.
    
    Коллекция изменяется при удалении/добавлении элементов или при изменении
    одной из входящих в коллекцию ссылок, например, когда одной из ссылок
    присваивается новое значение. В этом случае в соответствующих методах или
    свойствах класса бросаются события.
    
    При изменении данных объектов, ссылки на которые входят в коллекцию,
    значения самих ссылок не изменяются. Этот тип изменений непорождает событий.
    Для событий, извещающих об изменениях в коллекции, определяется свой
    делегат. События регистрируются в специальных классах-слушателях.
    
    Для событий предусмотреть возможность подписки и отписки отсобытия.
    
    Для обработки всех ошибочных ситуаций использовать конструкцию
    try...catch().
    
    В Main создать два экземпляра шаблонного класса-контейнера для разных типов
    данных. Работа с этими объектами должна демонстрироваться на следующих
    операциях:
    
    \begin{itemize}
        \item добавить;
        \item просмотреть;
        \item найти;
        \item удалить;
        \item найти;
        \item просмотреть.
    \end{itemize}

    Отладить и выполнить полученную программу. Проверить обработку
    исключительных ситуаций (например, чтение из пустого стека, дублирование
    объектов и т.п.).

    \makelisting{1.cs}

    \makelisting{1.txt}[Вывод программы]

    \textbf{Вывод}: освоены делегаты и события в языке \csharp.

\end{document}
