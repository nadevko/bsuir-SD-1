\documentclass{bsuir}

\departmentlong{инженерной психологии и эргономики}
\workcode{11}
\worktitle{Разработка отладка и испытание программ\\с пользовательским интерфейсом WPF}
\titleleft{
    Проверил:\\
    Давыдович К.И.\\
    ~
}
\titleright{
    Выполнил:\\
    Бородин А.Н.\\
    гр. 310901
}
\titlepageyear{2024}

\newfontface\liberationrm{Liberation Serif}
\newcommand{\csharp}{C{\liberationrm\#}}

\begin{document}

\maketitle
\mainmatter
\renewcommand{\thefigure}{\arabic{figure}}
\renewcommand{\thelisting}{\arabic{listing}}

\textbf{Цель}: Сформировать навыки разработки программ с использованием Windows
Presentation Foundation.

\section*{Задание 2}

Создайте список, в котором элементы могут изменять начертание символов,
например, с обычного начертания на курсивное. При клике на элемент списка
выделяется только этот элемент, но не отменяется начертание остальных элементов.

\makeimage[Окно программы]{1.png}[width=.5\textwidth]

\makelisting{App.xaml}[App.xaml]

\makelisting{App.xaml.cs}[App.xaml.cs]

\makelisting{MainWindow.xaml}[MainWindow.xaml]

\makelisting{MainWindow.xaml.cs}[MainWindow.xaml.cs]

\textbf{Вывод}: получены навыки и освоена разработка программ с использованием
Windows Presentation Foundation.

\end{document}
