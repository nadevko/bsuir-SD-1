\documentclass{bsuir}

\departmentlong{инженерной психологии и эргономики}
\workcode{1}
\worktitle{Разработка, отладка и испытание программ, основанных\\на операторах
ветвления и циклических алгоритмах}
\titlepageyear{2024}

\newfontface\liberationrm{Liberation Serif}
\newcommand{\csharp}{C{\liberationrm\#}}

\begin{document}

\maketitle{
    Проверил:\\
    Давыдович К.И.\\
    ~
}{
    Выполнил:\\
    Бородин А.Н.\\
    гр. 310901l
}

\textbf{Цель}: разрабатывать программы с применением ввода-вывода информации на
экран, условного оператора, оператора множественного выбора и циклов на языке
программирования \csharp.

\section*{Задание 1.2}

Товар стоит a руб. b коп. За него заплатили c руб. d коп. Сколько сдачи
требуется получить? Вводятся 4 числа: a, b, c и d. Необходимо вывести 2 числа в
виде: «Сдача составляет e рублей, f копеек».

\makelisting{1.cs}

\makelisting{1.txt}[Вывод программы (нулевой остаток)]

\makelisting{2.txt}[Вывод программы (копейки)]

\makelisting{2.txt}[Вывод программы (задолженность)]

\textbf{Вывод}: освоен минимальный набор конструкции языка \csharp.

\end{document}
