\documentclass{bsuir}

\departmentlong{инженерной психологии и эргономики}
\workcode{4}
\worktitle{ООП. Наследование. Абстрактные классы и члены классов. Виртуальные
члены классов. Запечатанные классы и члены классов}
\titlepageyear{2024}

\newfontface\liberationrm{Liberation Serif}
\newcommand{\csharp}{C{\liberationrm\#}}

\begin{document}

    \maketitle{
        Проверил:\\
        Давыдович К.И.\\
        ~
    }{
        Выполнил:\\
        Бородин А.Н.\\
        гр. 310901
    }

    \textbf{Цель}: получить навыки создания абстрактных классов и членов класса,
    создания виртуальных членов классов. Изучить наследование и основные принципы
    ООП.

    \section*{Задание 4.2}

    По полученному базовому классу (из предыдущей лабораторной работы) создать
    классы наследников по двум разным ветвям наследования (B←P1←P11 и B←P2←P21):
    
    \begin{itemize}
        \item во всех классах должны быть свои данные (характеристики объектов);
        \item во всех классах создать конструкторы инициализации объектов для
              всех классов (не забыть про передачу параметров в конструкции
              базовых классов);
        \item остальные методы создавать по необходимости.
    \end{itemize}

    Создать в базовом классе виртуальные функции расчета (например, расчет
    площади фигуры и т.п.) и вывода объекта на экран (всех его параметров).
    Выполнить реализацию этих виртуальных функций в классах наследниках.
    
    В классе контейнере создать массив, состоящий из объектов базового класса.
    Заполнить массив динамически создаваемыми объектами производных классов (P1,
    P11, P2, P21). Для каждого элемента массива проверить работу виртуальных
    функций.
    
    Отладить и выполнить полученную программу. Проверить, что будет, если
    вышеописанные методы не будут виртуальными.

    \makelisting{1.cs}

    \makelisting{1.txt}[Вывод программы]

    \textbf{Вывод}: освоены абстрактные классы и основные принципы ООП в языке
    \csharp.

\end{document}
