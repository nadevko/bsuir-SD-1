\documentclass{bsuir}

\departmentlong{инженерной психологии и эргономики}
\workcode{7}
\worktitle{Интерфейсы. Коллекции}
\titleleft{
    Проверил:\\
    Давыдович К.И.\\
    ~
}
\titleright{
    Выполнил:\\
    Бородин А.Н.\\
    гр. 310901l
}
\titlepageyear{2024}

\newfontface\liberationrm{Liberation Serif}
\newcommand{\csharp}{C{\liberationrm\#}}

\begin{document}

    \maketitle

    \textbf{Цель}: получить навыки создания и реализации интерфейсов. Изучить
    стандартные коллекции языка \csharp.

    \section*{Задание 7.2}

    Реализовать интерфейс, который будет расширять предметную область из
    предыдущих лабораторных работ. Добавить в интерфейс:
  
    \begin{itemize}
        \item метод вывода на экран всех полей класса;
        \item свойство содержащее название объекта;
        \item метод, который делает реверс названия объекта.
    \end{itemize}
    
    Добавить в интерфейс для одного из методов реализацию по умолчанию.
  
    Реализовать интерфейс всеми классами потомками, которые находятся в самом
    низу иерархии наследования.
  
    Добавить в программу любой класс, который будет реализовывать интерфейс.
  
    В главном классе создать метод, который будет создавать экземпляр класса из
    стандартной библиотеки коллекции, в качестве типа хранимых объектов
    использовать тип интерфейса.
  
    Вывести на экран все элементы коллекции, затем сделать реверс названий
    объектов и повторить операцию.
  
    Для обработки всех ошибочных ситуаций использовать конструкцию
    try...catch().
  
    \makelisting{1.cs}

    \makelisting{1.txt}[Вывод программы]

    \textbf{Вывод}: освоены интерфейсы и стандартные коллекции языка \csharp.

\end{document}
