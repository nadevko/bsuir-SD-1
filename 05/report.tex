\documentclass{bsuir}

\departmentlong{инженерной психологии и эргономики}
\workcode{5}
\worktitle{Обобщения и шаблоны}
\titleleft{
    Проверил:\\
    Давыдович К.И.\\
    ~
}
\titleright{
    Выполнил:\\
    Бородин А.Н.\\
    гр. 310901
}
\titlepageyear{2024}

\newfontface\liberationrm{Liberation Serif}
\newcommand{\csharp}{C{\liberationrm\#}}

\begin{document}

\maketitle
\mainmatter
\renewcommand{\thefigure}{\arabic{figure}}
\renewcommand{\thelisting}{\arabic{listing}}

\textbf{Цель}: получить навыки создания абстрактных классов и членов класса,
создания виртуальных членов классов. Изучить наследование и основные принципы
ООП.

\section*{Задание (2)}

Требуется построить контейнерный шаблонный класс (очередь) операций над
элементами данных, включающий операции:

\begin{itemize}
    \item добавления;
    \item удаления;
    \item поиска;
    \item просмотра;
    \item сортировки элементов;
    \item перестановки элементов в обратном порядке;
    \item замены всех подобных элементов по заданному ключу;
    \item поиска максимального элемента.
\end{itemize}

Остальные функции добавлять по необходимости.

\makelisting{Program.cs}

\makelisting{1.txt}[Вывод программы]

\textbf{Вывод}: освоены шаблоны и основные принципы ООП в языке \csharp.

\end{document}
